\documentclass{article}

% if you need to pass options to natbib, use, e.g.:
%     \PassOptionsToPackage{numbers, compress}{natbib}
% before loading neurips_2021

% ready for submission
\usepackage[preprint]{neurips_2021}

% to compile a preprint version, e.g., for submission to arXiv, add add the
% [preprint] option:
%     \usepackage[preprint]{neurips_2021}

% to compile a camera-ready version, add the [final] option, e.g.:
%     \usepackage[final]{neurips_2021}

% to avoid loading the natbib package, add option nonatbib:
%    \usepackage[nonatbib]{neurips_2021}

\usepackage[utf8]{inputenc} % allow utf-8 input
\usepackage[T1]{fontenc}    % use 8-bit T1 fonts
\usepackage{hyperref}       % hyperlinks
\usepackage{url}            % simple URL typesetting
\usepackage{booktabs}       % professional-quality tables
\usepackage{amsfonts}       % blackboard math symbols
\usepackage{nicefrac}       % compact symbols for 1/2, etc.
\usepackage{microtype}      % microtypography
\usepackage{xcolor}         % colors
\usepackage{authblk}


\title{SARSA on FrozenLake}

% The \author macro works with any number of authors. There are two commands
% used to separate the names and addresses of multiple authors: \And and \AND.
%
% Using \And between authors leaves it to LaTeX to determine where to break the
% lines. Using \AND forces a line break at that point. So, if LaTeX puts 3 of 4
% authors names on the first line, and the last on the second line, try using
% \AND instead of \And before the third author name.


\author{%
Ugo Marchesini \\
\texttt{ugo.marchesini@studio.unibo.it}
}



\begin{document}

\maketitle

\begin{abstract}
	
\textbf{FrozenLake} is a game provided by the Gym library to practice the methods of \textbf{Reinforced Learning}.
Gym provides APIs to act on the environment with an action and return the observable status and reward value.
The aim of this project is to apply the \textbf{SARSA} method to the \textbf{FrozenLake} game and experiment with some variations
	
\end{abstract}

\section{Problem}

FrozenLake is a GridWorld type game where the character can move in 4 directions. If it enters a hole square the episode ends with a reward of 0, if it enters a goal square the episode ends with a reward of +1.
If the argument is\_slippery = True then the character has a \nicefrac{1}{3} chance of completing the action in the direction set by the action, \nicefrac{1}{3} chance of performing the action orthogonal to the direction set by the action, this for both directions.
There are no rewards for other actions or reaching other squares.
The map set for this project is 4X4, but it is customizable.


\section{Solution}

The chosen solution is SARSA because the size of the observable space which is given by the number of cells (4X4) and the size of the possible actions (4) is such as to use a tabular method. In case the state space is too high, a solution with an approximation function is the only way to go.
The choice of the action is done in this project in 2 ways, the first through the $\epsilon$-greedy and the second through the \textbf{decreasing} $\epsilon$.
The second action selection mode is $\epsilon$-greedy decreasing , i.e. each time the goal is reached $\epsilon$ is reduced by a deltaepsilon (10*10\textsuperscript{-6}).

When epsilon (10*10\textsuperscript{-2}) reaches the minimum value of deltaepsilon the selection is almost greedy.
The use of a decreasing $\epsilon$ is justified by the convergence of the policy to determinism as the episodes continue.

\section{Result}

Initializing the QValue array with random values does not lead to convergence, as reported in the table below

Initializing the QValue matrix to 0 and using a fixed $\epsilon$ converges quite quickly with a maximum performance of 15\% of goals achieved on episodes.

Initializing the QValue matrix to 0 and using a decreasing $\epsilon$ converges faster with a maximum performance of 72\% goals achieved on episodes.

\begin{center}
\begin{tabular}{||c c c c||}
\hline
 Episode & Random & $\epsilon$-0 & Decr-0 \\ [0.5ex] 
\hline\hline
0 &   0 &   0 &   0 \\
1 &   402 &   885 &   0 \\
2 &   423 &   1466 &   859 \\
3 &   432 &   1523 &   1700 \\
4 &   407 &   1418 &   2056 \\
5 &   501 &   1527 &   2682 \\
6 &   482 &   1516 &   4664 \\
7 &   460 &   1518 &   6366 \\
8 &   435 &   1531 &   5462 \\
9 &   465 &   1467 &   5775 \\
10 &   410 &   1583 &   6701 \\
11 &   422 &   1523 &   5216 \\
12 &   400 &   1487 &   4889 \\
13 &   492 &   1520 &   6058 \\
14 &   399 &   1576 &   7284 \\
15 &   441 &   1469 &   5636 \\
16 &   434 &   1498 &   6785 \\
17 &   448 &   1466 &   7157 \\
18 &   394 &   1497 &   5659 \\
19 &   384 &   1480 &   7258 \\
20 &   514 &   1466 &   5261 \\
\hline\hline
\end{tabular}
\end{center}


Every line are 10000 episodes.



\end{document}